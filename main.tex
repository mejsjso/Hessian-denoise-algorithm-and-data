\documentclass[12pt,a4paper]{article}
\usepackage[colorlinks=true, urlcolor=blue]{hyperref}
\usepackage{fontsize}
\usepackage{caption} % 确保有
\usepackage{subcaption} % 最好添加这个宏包来处理子图
% 中文支持
\usepackage[UTF8]{ctex}
% 页面布局
\usepackage[top=2.5cm, bottom=2.5cm, left=3cm, right=2.5cm]{geometry}
% 数学公式
\usepackage{amsmath, amssymb, amsthm}
% 图形
\usepackage{graphicx}
% 表格
\usepackage{booktabs}
% 算法
\usepackage{algorithm, algorithmic}
% 代码排版
\usepackage{listings}
% 超链接
\usepackage{hyperref}
% 参考文献管理
\usepackage{natbib}
% 强制图片位置所需的宏包
\usepackage{float}

\usepackage{cases}
\usepackage{listings}
\usepackage{xcolor}

% 设置代码样式
\lstset{
    basicstyle=\ttfamily\small,
    breaklines=true,
    frame=single,
    numbers=left,
    numberstyle=\tiny,
    keywordstyle=\color{blue},
    commentstyle=\color{gray},
}

% 设置超链接样式
\hypersetup{
    colorlinks=true,
    linkcolor=black,
    citecolor=red,
    urlcolor=cyan,
}

% 定义新命令
\newcommand{\code}[1]{\texttt{#1}}
\newcommand{\todo}[1]{\textcolor{red}{[TODO: #1]}}

% 标题信息
\title{\textbf{基于二次正则凸优化的噪声滤过技术}}

% =============================================
% **** 核心修改部分:作者信息 (更紧凑,添加双剑号标记) ****
% =============================================
\author{
    朱凯乐\footnotemark[1]\footnotemark[2], 
    姜博文\footnotemark[1]\footnotemark[2], 
    陈柳萱\footnotemark[1]\footnotemark[2], 
    林瑜昕\footnotemark[1]\footnotemark[2]
}
\date{\today}

\begin{document}

% 封面
\maketitle

% =============================================
% **** 核心修改部分:脚注机构和贡献声明的顺序 ****
% 脚注标记:
% \footnotemark[2] 对应同等贡献 (第一个脚注,编号 1)
% \footnotemark[1] 对应北京大学 (第二个脚注,编号 2)
% =============================================
\footnotetext[1]{这些作者对本文贡献相同} % 编号 [1]:同等贡献
\footnotetext[2]{北京大学} % 编号 [2]:北京大学

% 摘要 - 修改标题格式为左对齐并加粗
\noindent % 取消缩进
{\textbf{摘要}} % 增大字号并加粗

本文提出一种基于二阶正则化的噪声滤除方法。该方法基于真实信号通常具有二阶平滑特性,而噪声由于其随机性会导致二阶导数在某些位置出现显著峰值的假设。通过引入保真项和二阶导数惩罚项构建二次规划问题,并采用分裂布雷格曼方法(一种正则化凸优化技术)进行求解,在保持信号真实性的同时有效滤除噪声。\cite{jia2018fast}实验结果表明,该噪声滤除技术能够在保持图像细节的前提下,显著降低噪声水平与强度,达到较好的噪声滤过效果。

% 关键词
\vspace{1em}
\noindent\textbf{关键词:} 二阶惩罚;噪声滤过;分裂布雷格曼方法

% 恢复正常的段落格式
\setlength{\parindent}{2em}
\setlength{\parskip}{0ex}
\normalsize\normalfont


% 正文开始
\section{引言}

\subsection{研究背景}

在工程应用中,我们经常遇到具有二阶平滑特性的信号(如生物医学成像等领域的图像)。然而,这些信号往往会由于受到噪声干扰而具有较低的信噪比。为了提高信号质量,我们需要开发有效的噪声滤除技术提升信噪比,以适应工程应用需求。

\subsection{研究目的}

本文旨在说明一种系统性方法,其能够实现对二阶光滑分布的图像的信噪比的提高。我们会将出一个通用且有效的算法,来改善此类图像的质量。

\subsection{相关方法及优缺点分析}

除本文方法外,现有图像增强技术包括高斯卷积核滤波和全变分(TV)滤波等。然而,这些方法存在一定局限性:高斯模糊仅对噪声进行平均处理,并未完全消除噪声,且在低信噪比条件下易导致图像失真。而TV滤波基于一阶导数惩罚,重建图像仅能实现分段光滑,而在不同区域间会出现块状伪影。\cite{jia2018fast}

本文介绍的方法(基于二阶导数惩罚)则能够在一定程度上实现对噪声的去除,并且由于其基于二阶惩罚,我们滤波的结果将是全局光滑的,重建图像不会呈现斑块状。\cite{jia2018fast}

\section{模型与算法}

\subsection{滤波的数学模型与求解}

\subsubsection{滤波的数学模型}

我们基于前述的二阶光滑假设构造如下拟真项与惩罚项:\cite{jia2018fast}

拟真项:$\frac{\mu}{2}\left\|g-f\right\|_{2}^{2}=\iint \frac{\mu}{2}(g-f)^2 dxdy$\footnote{这里我们将含噪声的原始图像的灰度分布记为$g$,将我们想要的、待求的去噪图像的灰度分布记为$f$,并将f关于对应方向的二阶导数记为$f_{xx},f_{xy},f_{yy}$}

二阶惩罚项:$\iint \left\|\begin{matrix}f_{xx}&f_{xy}\\f_{yx}&f_{yy}\end{matrix}\right\|_1 dxdy=\iint (\left\|f_{xx}\right\|_1+2\left\|f_{xy}\right\|_1+\left\|f_{yy}\right\|_1) dxdy$

我们希望我们最终的重建图像能够与含噪图像较为接近,也即我们希望拟真项不应该太大。同时,根据二阶光滑性,我们希望即二阶惩罚项也不能太大。因此,目标函数可表述为以下优化问题:
\begin{equation}
arg\ min_{f} [\frac{\mu}{2}\left\|g-f\right\|_{2}^{2}+\iint \left\|\begin{matrix}f_{xx}&f_{xy}\\f_{yx}&f_{yy}\end{matrix}\right\|_1 dxdy]
\end{equation}

求解这个二次规划问题需要大量的数学分析,我们下面来逐一分析:

\subsubsection{引入拉格朗日未定乘子}

二次规划问题:$arg\ min_{f} [\frac{\mu}{2}\left\|g-f\right\|_{2}^{2}+\iint (\left\|f_{xx}\right\|_1+2\left\|f_{xy}\right\|_1+\left\|f_{yy}\right\|_1) dxdy]$存在强耦合,非常不利于求解。为解耦优化问题,我们引入辅助变量:

\begin{equation}
d_{xx} = f_{xx}\ ,\ d_{xy} = 2f_{xy}\ ,\ d_{yy} = f_{yy}
\end{equation}

从而上述函数等于$ \frac{\mu}{2}\left\|g-f\right\|_{2}^{2}+\iint (\left\|d_{xx}\right\|_1+\left\|d_{xy}\right\|_1+\left\|d_{yy}\right\|_1) dxdy$。采取增广拉格朗日函数方法,引入未定乘子以实现彻底脱耦合,最终问题化为:
\begin{align}
arg\ min_{f,d}[\frac{\mu}{2}\left\|g-f\right\|_{2}^{2}+\iint (\left\|d_{xx}\right\|_1+\left\|d_{xy}\right\|_1+\left\|d_{yy}\right\|_1) dxdy+  \nonumber \\
 \frac{\lambda}{2}(\left\|d_{xx}-f_{xx}\right\|_{2}^{2}+\left\|d_{xy}-2f_{xy}\right\|_{2}^{2}+\left\|d_{yy}-f_{yy}\right\|_{2}^{2})]
\end{align}

我们通过引入拉格朗日乘子,实现了对前述函数的解耦。但是,直接求解这个函数的极值条件仍然十分具有难度,故我们下面利用分裂布雷格曼迭代法(Split-Bregman迭代法)对极值条件进行求解:

\subsubsection{分裂布雷格曼迭代法求解极值条件}

引入迭代参量$b_{xx},b_{xy},b_{yy}$,优化问题改写为:
\begin{align}
arg\ min_{f,d}[\frac{\mu}{2}\left\|g-f\right\|_{2}^{2}+\iint (\left\|d_{xx}\right\|_1+\left\|d_{xy}\right\|_1+\left\|d_{yy}\right\|_1) dxdy+  \nonumber \\
 \frac{\lambda}{2}(\left\|d_{xx}-f_{xx}-b_{xx}\right\|_{2}^{2}+\left\|d_{xy}-2f_{xy}-b_{xy}\right\|_{2}^{2}+\left\|d_{yy}-f_{yy}-b_{yy}\right\|_{2}^{2})]
\end{align}

\textbf{下面对这个方程进行迭代求解:}

我们将
$\frac{\lambda}{2}(\left\|d_{xx}-f_{xx}-b_{xx}\right\|_{2}^{2}+ 
\left\|d_{xy}-2f_{xy}-b_{xy}\right\|_{2}^{2}+ \left\|d_{yy}-f_{yy}
\\-b_{yy}\right\|_{2}^{2})$
简记为$\frac{\lambda}{2}(\sum_i \left\|d_i-\nabla_i f - b_i \right\|_2^2)$,(其中$i\in\{xx,xy,yy\}$且$\nabla_{xx}=\partial_{xx}^2,\nabla_{xy}=2\partial_{xy}^2,\nabla_{yy}=\partial_{yy}^2$),则前述问题可以改写为:
\begin{align}
&arg\ min_{f,d}[\frac{\mu}{2}\left\|g-f\right\|_{2}^{2}+\iint (\left\|d_{xx}\right\|_1+\left\|d_{xy}\right\|_1+\left\|d_{yy}\right\|_1) dxdy+  \nonumber \\
&\frac{\lambda}{2}(\sum_i \left\|d_i-\nabla_i f - b_i \right\|_2^2)]
\end{align}


\textbf{先固定$b$,$d$求解f:}

将$(5)$式中的函数对$f$求导,这是一个广泛函数,求导过程如下所示(我们采用泛函里面经典的变分法进行求解):
\begin{align}
f=f+\delta f
\end{align}
\begin{align}
\delta(\frac{\mu}{2}\left\|g-f\right\|_2^2 )= \frac{\mu}{2}\iint(g-f-\delta f)^2dxdy-\frac{\mu}{2}\iint(g-f)^2dxdy =\mu \iint (f-g)\delta f dxdy
\end{align}

根据泛函导数的定义:
\begin{align}
\partial_f \frac{\mu}{2}\left\|g-f\right\|_2^2 = \mu(f-g)
\end{align}

同理考虑后面几项的导数:
\begin{align}
\delta(\frac{\lambda}{2}\left\|d_i-\nabla_i f_i-b_i\right\|_2^2) = - \lambda\iint (d_i-\nabla_i f_i-b_i)\nabla_{i}(\delta f) dxdy  
\end{align}

引入$\nabla_i$的伴随导数$\nabla_i^T$,则上式化为:
\begin{align}
\delta(\frac{\lambda}{2}\left\|d_i-\nabla_i f_i-b_i\right\|_2^2) = - \lambda\iint (\nabla_{i})^T(d_i-\nabla_i f_i-b_i)\delta f dxdy  
\end{align}

也即导数为:
\begin{align}
\partial_f(\frac{\lambda}{2}\left\|d_i-\nabla_i f_i-b_i\right\|_2^2) = - \lambda\nabla_{i}^T(d_i-\nabla_i f_i-b_i)
\end{align}

进而我们可以进一步地写出完整的求导结果:
\begin{align}
(\mu+\sum_i\lambda \nabla_i^T\nabla_i)f=(\mu g +\lambda\sum_i \nabla_i^T(d_i-b_i))
\end{align}

这个方程在空域内难以求解,我们将其转换到频域内进行求解,首先我们证明一个性质,即空域内的微分操作等价于频域内的乘法操作:
\begin{align}
\partial_x\iint f e^{ik_xx+ik_yy}dxdy = ik_x\cdot\iint f e^{ik_xx+ik_yy} dxdy
\end{align}

进而,我们可以有(下面的FFT代表快速傅里叶变换):

\begin{align}
(\mu+\sum_i \lambda FFT(\nabla_i^T\nabla_i))FFT(f) = (\mu FFT(g)+\lambda \sum_i FFT((\nabla_i^T(d_i-b_i)))
\end{align}

也即:
\begin{align}
f = FFT^{-1}(\frac{\mu FFT(g)+\lambda \sum_i FFT(\nabla_i^T(d_i-b_i))}{\mu+\sum_i \lambda FFT(\nabla_i^T\nabla_i})
\end{align}

上式容易推广为迭代形式:
\begin{align}
f^{(k+1)} = FFT^{-1}(\frac{\mu FFT(g)+\lambda \sum_i FFT(\nabla_i^T(d^{(k)}_i-b^{(k)}_i))}{\mu+\sum_i \lambda FFT(\nabla_i^T\nabla_i)})
\end{align}

\textbf{接下来我们固定$f$与$b$求解$d$,容易看出,各个$d_i$的优化问题是独立的,也即我们可以分别考虑:}
\begin{align}
arg\ min_{d_i}[\left\|d_i\right\|_1+\frac{\lambda}{2}\left\|d_i-\nabla_i f -b_i\right\|_2^2]
\end{align}

对上式求导并令$\nabla_i f - b_i=u_i$,可以得到:
\begin{align}
\frac{\partial\left\|d_i\right\|_1}{\partial d_i}+\lambda(d_i-u_i) =0
\end{align}
其中:
\begin{numcases}{}
\frac{\partial\left\|d_i\right\|_1}{\partial d_i} = 1\ \ \ (if\ \ d_i>0)\nonumber\\
\frac{\partial\left\|d_i\right\|_1}{\partial d_i} \in [-1,1]\ \ \ (if\ \ d_i=0)\\
\frac{\partial\left\|d_i\right\|_1}{\partial d_i} = -1 \ \ \ (if\ \ d_i<0)\nonumber
\end{numcases}

因此,当$d_i>0$时,$d_i=u_i-\frac{1}{\lambda}$,此时要求$u_i>\frac{1}{\lambda}$;当$d_i<0$时,$d_i=u_i+\frac{1}{\lambda}$,此时要求$u_i<-\frac{1}{\lambda}$;当$d_i=0$时,$d_i=u_i-\frac{1}{\lambda}\cdot\frac{\partial\left\|d_i\right\|_1}{\partial d_i}=0$,此时要求$u_i\in[-\frac{1}{\lambda},\frac{1}{\lambda}]$

根据凸优化的性质,可以认为我们给出了正确的递推函数\footnote{此处我们记$u_i^{(k)}=\nabla_i f^{(k+1)}-b_i^{(k)}$}

\begin{numcases}{}
d_i^{(k+1)} = u_i^{(k)}-\frac{1}{\lambda}\ \ \ (if\ \ u_i^{(k)}>\frac{1}{\lambda}) \nonumber \\
d_i^{(k+1)} = 0\ \ \ (if\ \ u_i^{(k)}\in[-\frac{1}{\lambda},\frac{1}{\lambda}])  \\
d_i^{(k+1)} = u_i^{(k)}+\frac{1}{\lambda}\ \ \ (if\ \ u_i^{(k)}<-\frac{1}{\lambda}) \nonumber 
\end{numcases}

也即$d_i$的递推可以由收缩算子$shrink$给出:
\begin{align}
d_i^{(k+1)}=shrink(u_i^{(k)},\frac{1}{\lambda})=sign(u_i^{k})\cdot max(|u_i^{(k)}|-\frac{1}{\lambda},0)
\end{align}

最后,我们给出分裂布雷格曼方法要求的统一的$b_{xx},b_{xy},b_{yy}$的迭代方法:
\begin{numcases}{}
b_{xx}^{(k+1)} = b_{xx}^{(k)} + f_{xx}^{k+1} - d_{xx}^{(k+1)} \nonumber\\
b_{xy}^{(k+1)} = b_{xy}^{(k)} + f_{xy}^{k+1} - d_{xy}^{(k+1)} \\
b_{yy}^{(k+1)} = b_{yy}^{(k)} + f_{yy}^{k+1} - d_{yy}^{(k+1)}\nonumber
\end{numcases}

通过迭代求解,根据分裂布雷格曼迭代方法的收敛性可知:算法最终收敛至目标解。

\section{具体算法}

我们基于北太天元平台实现了所提出的二阶正则化去噪算法。由于代码逾300行,未在文中完整展示。本文涉及的所有算法均已开源,详见:
\href{https://github.com/zhukaile5-crypto/Hessian-denoise-algorithm-and-data.git}{\url{https://github.com/zhukaile5-crypto/Hessian-denoise-algorithm-and-data.git}}(除了本文所讨论的主代码外,我们还在github上上传了空间高斯滤波代码用作对比,并上传了对原始图像(包括黑白的和彩色的)的加噪代码)。

\section{测试结果与分析}

\subsection{算法去噪性能的直观展示与定量分析}

\subsubsection{不同噪声幅度和占比条件下的去噪效果}

我们在不同噪声强度条件下测试算法性能,结果如图1所示:

\begin{figure}[H]
    \centering
    
    \begin{minipage}{0.75\linewidth}
        \centering
        \includegraphics[width=\linewidth]{不同噪声强度去噪结果.png}
        
        % 使用 \fontsize{大小}{行距}\selectfont 来设置 pt 字号
        \fontsize{7pt}{11pt}\selectfont 
        \vspace{-10pt}
        \caption{不同噪声强度下的去噪效果——第一、三行为噪声图像、第二、四行为去噪结果,第1、3行编码从1-6的图像的噪声方差为0,50,100,150,200,250,噪声图像正下方的图像为其对应的去噪图像}
        
        \label{fig:placeholder}
    \end{minipage}
    % 注意:在 minipage 结束后,字体大小会自动恢复。
\end{figure}

测试结果表示:不论噪声的方差取在$[0,255]$的任何一个区间,程序均有良好的降噪效果,为定量评估去噪性能,我们计算了结构相似性(SSIM)和峰值信噪比(PSNR)指标,结果如图2所示:
\begin{figure}[H]
    \centering
    \begin{minipage}{0.75\linewidth}
        \centering
        
        \includegraphics[width=\linewidth]{不同噪声强度去噪性能.png}
        \fontsize{7pt}{11pt}\selectfont 
        \vspace{-13pt}
        \caption{不同噪声强度下代码的去噪性能(图中展示了两个主要参数——SSIM的提升幅度(它可以反映算法对图像的复原度)以及PSNR(它可以展示图像的去噪性能))}
        \label{fig:placeholder}
    \end{minipage}
\end{figure}


我们想知道算法在高噪声下的表现,故测试了不同噪声密度(5\%,10\%,15\%,20\%,25\%)下的去噪效果,结果如下图所示:
\begin{figure}[H]
    \centering
    \begin{minipage}{0.9\linewidth}
        \centering
    \includegraphics[width=0.75\linewidth]{不同噪声占比去噪结果.png}
    \fontsize{7pt}{11pt}\selectfont 
     \vspace{-10pt}
    \caption{不同噪声占比下的去噪效果——第一、三两行为噪声图像,第二、四两行为去噪结果}
    \label{fig:placeholder}


    \end{minipage}
\end{figure}

测试结果显示:代码在噪声占比很高且噪声强度较大的条件下仍然能够显出良好的去噪能力,图像经去噪后,原本呈现雪花状密集分布的噪声退化为较弱的背景虚化像,同时真实点状图像得到保留。

对应的定量评估结果如下图4所示:

\begin{figure}[H]
    \centering
     \begin{minipage}{0.9\linewidth}
        \centering
    \includegraphics[width=0.75\linewidth]{不同噪声占比去噪性能.png}
     \fontsize{7pt}{11pt}\selectfont 
     \vspace{-10pt}
    \caption{不同噪声占比下代码的去噪性能}
    \label{fig:placeholder}
    \end{minipage}
\end{figure}



\subsection{代码对噪声涨落的平滑效果}

我们的代码能够在很大程度上降低噪声的幅度,使得图像的灰度曲面更为平滑。通过绘制灰度-XY 曲线图,我们展现代码对噪声的滤除效果以及对图像的平滑作用。

\subsubsection{全局灰度曲面对比}

首先,我们先来看我们的代码对图像整体的噪声滤除作用与涨落平滑作用,相关结果如下图5所示:

\begin{figure}[H]
    \centering
    \makebox[\linewidth]{
        \includegraphics[width=1\linewidth]{yanshi2.png} 
    }
    \vspace{-10pt}
    \caption{原始-噪声-去噪图像的全局灰度曲面对比图}
    \label{fig:yanshi2}
\end{figure}
\vspace{-10pt}
图像表明,程序实现了对原噪声的高度抑制以及对噪声涨落的高度平滑,并且使得图像的灰度分布与原图像(真实图像)的接近程度大为提高。

\subsubsection{局部灰度曲面对比}

以下下展示代码在局部的去噪行为(如图6所示):

\begin{figure}[H]
    \centering
    \makebox[\linewidth]{
        \includegraphics[width=1\linewidth]{yanshi1.png} 
    }
    \vspace{-10pt}
    \caption{原始-噪声-去噪图像的局部灰度曲面对比图}
    \label{fig:yanshi2}
\end{figure}

图像结果表明,代码在局部同样具有良好的去噪(平滑)表现。

\subsection{与经典高斯算法的对比}

\subsubsection{定性的图像对比}

我们的算法是基于频域的去噪算法,而高斯模糊算法则是基于的空域的去噪算法,我们下面将两种算法进行对比。

首先,我们在极高噪声下考察两种算法的表现,具体结果如下图7所示:
\begin{figure}[H]
    \centering % 作用于整个图和图注块
    
    % 1. 放置放大的图片
    \includegraphics[width=0.7\linewidth]{结果展示.png}     
    \vspace{5pt} % 增加图片和图注之间的间距(可选)
    
    % 2. 放置图注块 (已移除干扰居中的 \makebox)
    \begin{minipage}{0.85\linewidth} 
        \centering % 居中 minipage 内部的标题文本
        \vspace{-10pt} % 保留您想要的间距调整
        
        % 放置图注
        \footnotesize\caption{我们在极高噪声的条件下用两种算法对图像进行去噪,图中共8张图片每行为一组照片,每组的四张照片依次为:原始图像、噪声图像、我们的代码的去噪结果、高斯模糊的去噪结果}
        \label{fig:yanshi2}
    \end{minipage}
\end{figure}
\vspace{-10pt}
图像结果显示:我们的频域算法重建出的图像更为平滑,且畸变性更弱。具体而言,我们算法的结果较为连续。而高斯模糊算法的结果则较为散乱,仍然呈现出点状散斑噪声的特性。此外,高斯算法重建出的图像有较为严重的畸变,原本的圆点在重建后无法保持原有形状,而我们的算法则能实现基本保形,即圆点在重建后至多变为椭球状。

下面我们从图像相似度(SSIM)提升与噪声去除率两个角度来定量衡量我们的算法相较于经典高斯算法的优势:

\subsubsection{SSIM提升角度}

两种算法对图像的相似度提升的对比结果如下图8所示:

\begin{figure}[H]
    \centering % 作用于整个图和图注块
    
    % 1. 放置图片
    \includegraphics[width=0.6\linewidth]{两种算法的相似度提升对比.png} 
    
    % 2. 放置图注块 (使用 minipage 限制宽度,并依靠 \centering 居中)
    \begin{minipage}{0.75\linewidth} 
        \centering % 居中 minipage 内部的标题文本
        \vspace{-10pt} % 保留您想要的间距调整
        
        % 放置图注
        \caption{两种算法的相似度提升对比(其中红线代表我们的高斯算法,蓝线代表经典的高斯模糊算法)}
        \label{fig:yanshi2}
    \end{minipage}
\end{figure}

上图的定量结果显示,我们的算法在相似度提升上相较于高斯算法有着30\%-40\%的恒定优势


\subsubsection{噪声去除率角度}

两种算法对图像的噪声去除率的对比结果如下图8所示:


\begin{figure}[H]
    \centering
    % 1. 图片部分:使用 \makebox 允许图片放大(例如 1.15\linewidth)并居中
    \makebox[\linewidth]{
        \includegraphics[width=0.6\linewidth]{两种算法的噪声去除率对比.png} 
    }
    
    % 2. 图注部分:使用 \makebox 居中图注的窄框
    \makebox[\linewidth]{
        % 使用一个窄的 minipage 来限制图注的宽度(例如 0.8\linewidth)
        \begin{minipage}{0.75\linewidth} 
            \centering % 居中 minipage 内部的标题文本
            \vspace{-10pt} % 保留您想要的间距调整
            
            % 放置图注
            \caption{两种算法的噪声去除率对比(其中红线代表我们的算法,蓝线代表高斯模糊算法)}
            \label{fig:yanshi2}
        \end{minipage}
    }
\end{figure}

噪声去除率对比图显示:绝大多数情况下,我们的算法相较于高斯模糊算法有着较为明显的优势。

此外,就算法灵活性而言,我们的代码的核心参数仅有$\mu$,需要保留原图像特征时调高$\mu$、需要尽可能去噪时调小$\mu$即可。而若要调整高斯模糊算法,则要改变卷积核的大小,这直接影响了分布标准差,需要进一步的调整。因此,相较于传统算法,我们的算法体现了高度的灵活性。

\subsection{算法的分辨率损伤分析}

许多去噪算法存在着一个关键问题:在去除噪声的过程中会降低分辨率,从而损失精细结构信息。此处我们进行有关分辨率损伤的分析:

在生成并重构标准正弦图像及其加噪图像后,我们分析正弦函数图像的分辨性,从而判定我们的代码是否存在分辨率损伤的问题,结果如下图9所示:

\begin{figure}[H]
    \centering
    \makebox[\linewidth]{
        \includegraphics[width=0.5\linewidth]{总结果.png} 
    }
    \vspace{-10pt}
    \caption{分辨率损伤分析图}
    \label{fig:yanshi2}
\end{figure}

需要申明的是:此结果是原图像的放大版本,我们实际设置的空间频率接近实际图像中可能达到的最高频率,即人眼已经无法分辨放大前的图像为正弦图像或均匀图像。在此前提下进行去噪,得到的最终结果显示:我们的算法并不会对此等空间频率的图像造成完全损伤。因此我们可以认为:我们的算法对图像分辨率的损伤完全处在可以容忍的范围之内。

\subsection{算法的具体应用与彩色延拓}

\subsubsection{算法的具体应用}

在生物学应用当中,我们常常需要对处在恒定地运动状态中的活细胞进行成像。因此,为了避免出现运动伪影,成像的频率会相当高(通常在1000Hz左右),这使得接收到的光子数极少,导致了很低的信噪比。此时需要运用滤噪算法处理图像,以下展示一个具体实例:

\begin{figure}[H]
    \centering
    \makebox[\linewidth]{
        \includegraphics[width=0.9\linewidth]{生物学去噪图.png} 
    }
    \vspace{-16pt}
    \footnotesize\caption{生物学去噪图}
    \label{fig:yanshi2}
\end{figure}
\vspace{-0.35cm}


\subsubsection{算法的彩色延拓}

我们指出,虽然去噪算法主要针对灰度图像的,但也可以通过处理RGB值而轻松地应用于彩色图像,相关地代码同样开源于github同一地址,下面展示了相关的去噪结果(声明:本部分第二张图来自于\cite{fieldhouse2025car}):

\begin{figure}[H]
    \centering
    
    % --- 第一张图 (上) ---
    \begin{minipage}{\linewidth}
        \centering
        \makebox[\linewidth]{
            \includegraphics[width=0.85\linewidth]{结果1.png} 
        }
        \label{fig:color_denoise_1} % 独立的子标签
    \end{minipage}
    
    % 大幅减小两张图之间的垂直间距
    \vspace{-1cm} % 示例值,您可以调整得更小
    
    % --- 第二张图 (下) ---
    \begin{minipage}{\linewidth}
        \centering
        \makebox[\linewidth]{
            \includegraphics[width=0.85\linewidth]{结果2.png} 
        }
        % 已经移除匕首脚注标记 \textsuperscript{\textdagger}
        \label{fig:color_denoise_2} % 独立的子标签
    \end{minipage}
    
    % --- 统一的图名 ---
    \vspace{-13pt} % 保留您原有的图注与图之间的间距调整
    \caption{彩色去噪结果}
    \label{fig:color_denoise_results} % 统一的引用标签
    
\end{figure}
\vspace{-0.75cm}
上述图像表明,我们的去噪程序可以有效地延拓至彩色图像处理领域。


\section{总结与展望}
由我们前面的相关讨论,我们可以得出论断:我们的算法相较于经典的空域高斯模糊算法有着非常明显的优势——我们在图像的相似度(SSIM)提升方面以及噪声的去除率(基于PSNR)方面相较于高斯模糊算法有着近乎恒定的优势。且我们的算法在可调性方面较高斯模糊算法更是有着极大的优势——也即我们的代码拥有高度的灵活性。这些优势肯定了它作为一种优秀的去噪算法。我们指出:它可以被运到生物医学成像等多个领域——自2019年来,该算法已经在超分辨结构光显微成像领域得到了广泛的应用,极大地提高了系统容许的噪声量与噪声幅度,使得我们能够以更短的时间、更低的光强进行单帧成像,进而实现了对生物样本的超快、低漂白、长时程的超分辨显微成像,为生物医学领域的研究做出了卓越的贡献!\cite{jia2018fast}\cite{chen2023superresolution}我们也希望此算法能够面向更广泛的大众得到普及,使得该算法能够被运用到更多尚未涉足的领域!


\clearpage % 强制换页

% 参考文献
\bibliographystyle{unsrt}
\bibliography{references}


\section{鸣谢}

\textbf{我们在此向为本文提供测试数据的Xilab表示诚挚的感谢,并在此向所有为本文的写作提供指导与帮助的老师和同学致以衷心的谢意!}\end{document}